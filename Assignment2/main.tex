\documentclass[conference]{IEEEtran}
\IEEEoverridecommandlockouts
\usepackage{listings}
\usepackage{xcolor}
\usepackage{hyperref}
\usepackage{amsthm}
\usepackage{pgfplots}
\usepackage{graphicx}
\lstset { %
    language=C++,
    backgroundcolor=\color{black!5}, % set backgroundcolor
    basicstyle=\footnotesize,% basic font setting
}
%New colors defined below
\definecolor{codegreen}{rgb}{0,0.6,0}
\definecolor{codegray}{rgb}{0.5,0.5,0.5}
\definecolor{codepurple}{rgb}{0.58,0,0.82}
\definecolor{backcolour}{rgb}{0.95,0.95,0.92}

%Code listing style named "mystyle"
\lstdefinestyle{mystyle}{
  backgroundcolor=\color{backcolour},   commentstyle=\color{codegreen},
  keywordstyle=\color{magenta},
  numberstyle=\tiny\color{codegray},
  stringstyle=\color{codepurple},
  basicstyle=\ttfamily\footnotesize,
  breakatwhitespace=false,         
  breaklines=true,                 
  captionpos=b,                    
  keepspaces=true,                 
  numbers=left,                    
  numbersep=5pt,                  
  showspaces=false,                
  showstringspaces=false,
  showtabs=false,                  
  tabsize=2
}

%"mystyle" code listing set
\lstset{style=mystyle}
% The preceding line is only needed to identify funding in the first footnote. If that is unneeded, please comment it out.
\usepackage{cite}
\usepackage{amsmath,amssymb,amsfonts}
\usepackage{algorithmic}
\usepackage{graphicx}
\usepackage{textcomp}
\usepackage{xcolor}
\def\BibTeX{{\rm B\kern-.05em{\sc i\kern-.025em b}\kern-.08em
    T\kern-.1667em\lower.7ex\hbox{E}\kern-.125emX}}
\begin{document}

\title{DAA Assignment-05\\
}

\author{\IEEEauthorblockN{ Nischay Nagar }
\IEEEauthorblockA{\textit{IIT2019198} \\
}
\and
\IEEEauthorblockN{Abhishek Bithu}
\IEEEauthorblockA{\textit{IIT2019199} \\
}
\and
\IEEEauthorblockN{Raj Chandra}
\IEEEauthorblockA{\textit{IIT2019200} \\
}
}

\maketitle

\begin{abstract}
In this report we designed a Dynamic Programming algorithm to find the propablity of a knight to remain in the cheessboard after k number of steps.

\end{abstract}

\section{Introduction}
It can be observed that at every step the Knight has 8 choices to choose from. Suppose, the Knight has to take k steps and after taking the Kth step the knight reaches (x,y). There are 8 different positions from where the Knight can reach to (x,y) in one step, and they are: (x+1,y+2), (x+2,y+1), (x+2,y-1), (x+1,y-2), (x-1,y-2), (x-2,y-1), (x-2,y+1), (x-1,y+2).
If the probablities of reaching these 8 positions after k-1 steps is already known then, the final probablity after k steps will simply be equal to the (Σ probability of reaching each of these 8 positions after K-1 steps)/8.

Here we are dividing by 8 because each of these 8 positions has 8 choices and position (x,y) is one of the choices.

For the positions that lie outside the board, we will either take their probabilities as 0 or simply neglect it.

Since we need to keep track of the probabilities at each position for every number of steps, we need Dynamic Programming to solve this problem.
We are going to take an array dp[x][y][steps] which will store the probability of reaching (x,y) after (steps) number of moves.

Base case: if the number of steps is 0, then the probability that the Knight will remain inside the board is 1

\section{Algorithm Design}
Following is Dynamic Programming algorithm

\begin{itemize}
\item Define direction vectors for the knight. 
\item Take an array dp[N, N, steps + 1] which will store the probability of reaching (x,y) after (steps) number of moves.
\item Base case: if the number of steps is 0, then the probability that the Knight will remain inside the board is 1
\item Take the position (x, y) after s number of steps.
\item Take prob = 0.0 then check for each position reachable from (x, y) using the direction vectors and store it in a new position (nx, ny).
\item Check if this new position (nx, ny) is inside of the chessboard, if yes then add dp1[nx][ny][s - 1] / 8.0 to prob.
\item Store the prob in dp[x][y][s].
\item Keep repeating for the given number of steps.
\item The required probability will be stored in dp[start_x][start_y][k], where (start_x, start_y) are the given initial position of the knight and k is the number of steps.


\section{Algorithm Analysis}
\textbf{Time complexity:}\\

O(NxNxK), where N is the size of the board and K is the number of steps
\\
\textbf{Space complexity:}\\

O(NxNxK), where N is the size of the board and K is the number of steps

\section{Conclusion}
We can observe that by Dynamic Programming Approach we can easily solve the given problem and get better time-complexity.

 \section{References}
\color{blue}1.{\url{https://www.geeksforgeeks.org/dynamic-programming/} }\\


\color{black}
\
\begin{titlepage}
    \begin{center}
        \Huge
        \section*{Appendix}
        \end{center}
         \textbf{Code for implementation of this paper is given below:}
\begin{lstlisting}[language=C++,caption=Code for this paper]
#include <bits/stdc++.h>
using namespace std;

int N = 0;

int dx[] = {1, 2, 2, 1, -1, -2, -2, -1};
int dy[] = {2, 1, -1, -2, -2, -1, 1, 2};

bool inside(int x, int y)
{
    return (x >= 0 and x < N and y >= 0 and y < N);
}

double findProb(int start_x, int start_y, int steps)
{
    double dp[N][N][steps + 1];

    for (int i = 0; i < N; ++i)
        for (int j = 0; j < N; ++j)
            dp[i][j][0] = 1;

    for (int s = 1; s <= steps; ++s)
    {

        for (int x = 0; x < N; ++x)
        {
            for (int y = 0; y < N; ++y)
            {
                double prob = 0.0;

                for (int i = 0; i < 8; ++i)
                {
                    int nx = x + dx[i];
                    int ny = y + dy[i];

                    if (inside(nx, ny))
                        prob += dp[nx][ny][s - 1] / 8.0;
                }

                dp[x][y][s] = prob;
            }
        }
    }

    return dp[start_x][start_y][steps];
}

int main()
{
    int K;
    int x, y;
    cout << "Enter the size of chessboard ";
    cin >> N;
    cout << "Enter the number of steps ";
    cin >> K;
    cout << "Enter the space-separated position of knight ";
    cin >> x >> y;
    cout << findProb(x, y, K) << endl;

    return 0;
}

   
\end{lstlisting}
\end{titlepage}
\end{document}
